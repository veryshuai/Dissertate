% the abstract

Each of the three chapters of this dissertation is based on an empirical research paper.  While the topic of each chapter is different, they are linked by methodology.  Each chapter develops a structural model of economic interaction, applies econometric techniques to estimate model parameters from data, and then uses the estimated model to analyze policy questions.

In the first chapter, I modify a recent theoretical model of conspicuous consumption to empirically measure the the importance of peer beliefs to Americans and Chinese.  Estimating the model on household budget surveys, I find that Chinese consumers care 20\% more than American consumers about peer  beliefs.  I use the estimated model to evaluate the welfare effect of the 1990-2002 American luxury tax on automobiles.  The luxury tax benefitted nearly all Americans a small amount, but hurt the small fraction of consumers who love automobiles the most.

The second chapter, a joint work with my adviser and others, seeks to understand the way Colombian and American firms interact in U.S. Customs data.  After documenting patterns in the data, we develop an estimable empirical model in which heterogeneous sellers engage in costly search for buyers.  Through meeting buyers, a firm gradually learns about the appeal of its product in the market, which affects its incentive to search for more buyers.  Fit using indirect inference, the model both replicates key patterns in the customs data and allows us to quantify several types of trade costs, including the search costs of\ identifying potential clients and the costs of maintaining business relationships with existing clients. We also estimate the effect of previous exporting activity on the costs of meeting new clients, and to characterize the cumulative effects of learning on a firm's search intensity. Finally, we use our fitted model to explore the effects of these trade costs and learning effects on aggregate export dynamics.

In the third chapter
