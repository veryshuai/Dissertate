% the abstract

Each of the three chapters of this dissertation is based on an empirical research paper.  While the topic of each chapter is different, they are linked by methodology.  Each chapter develops a structural model of economic interaction, applies econometric techniques to estimate model parameters from data, and then uses the estimated model for policy analysis.

In the first chapter, I modify a recent theoretical model of conspicuous consumption to empirically measure the the importance of peer beliefs to Americans and Chinese.  In the model, a consumer cares not only about the direct utility she receives from consumption, but also about the way her consumption pattern affects her peer group's belief about her well-being.  I estimate the model on household budget surveys using an EM algorithm. According to model estimates, a Chinese consumer cares 20\% more than an American consumer about peer beliefs.  I use the estimated model to evaluate the welfare effect of the 1990-2002 American luxury tax on automobiles.  The luxury tax benefitted nearly all Americans a small amount, but hurt the small fraction of consumers who love automobiles the most.

The second chapter, a joint work with my adviser and others, seeks to understand the way Colombian and American firms interact in U.S. Customs data.  After documenting patterns in the data, we develop an estimable empirical model in which heterogeneous sellers engage in costly search for buyers.  Through meeting buyers, a firm gradually learns about the appeal of its product in the market, which affects its incentive to search for more buyers.  Fit using indirect inference, the model both replicates key patterns in the customs data and allows us to quantify several types of trade costs, including the search costs of\ identifying potential clients and the costs of maintaining business relationships with existing clients. We also estimate the effect of previous exporting activity on the costs of meeting new clients, and to characterize the cumulative effects of learning on a firm's search intensity. Finally, we use our fitted model to explore the effects of these trade costs and learning effects on aggregate export dynamics.

The third chapter measures the effect of mobility between workplaces on the speed at which new ideas diffuse.  Using a new panel data set linking academics to departments and citations, I develop and estimate a dynamic model of location choice in which an idea is more likely to be encountered when colleagues already know about it.  Several exercises indicate that coworker knowledge significantly affects the probability of learning about a new idea.  Counterfactual exercises show that labor mobility increases the speed at which new ideas spread between locations, makes locations more uniform in the fraction of people who know about a new idea, and raises the percentage of people who know about a new idea at a given time.  A calibration using results from my baseline estimation indicates that international movement of workers can have a large effect on diffusion of knowledge into a developing country.

